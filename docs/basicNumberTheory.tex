
% !TEX root = DesignDocument.tex

\chapter{Basic Number Theory}

\section{cryptomath Library}

Included with this package is a basic Cryptography math library. 
This library takes advantage of the GMP (GNU Multi-Precision library) for arbitrarily large integers.
GMP provides many helpful functions to cryptographic programs. 
Some of them has be reimplemented here, most however are just useful functions.

Since GMP integers are represened by pointers to structures of memory there are almost no valued functions in this library.
Return values are almost always returned as the first argument.

Since most functions are implemented fully with GMP integers most of them will function with numbers as big as your computer can hold.
Whether or not the functions will finish is up to your own patience and contemplation on the halting problem.
However some are not implemented with GMP for various reasons.
These functions will be pointed out and reasons given as to why they were not implemented with GMP.
Bounds will also be given for non-GMP funtions since no bounds checking is done by the libary.

\section{Generic Function}
\subsection{Greatest Common Divisor} 
\begin{verbatim}
gcd( mpz_t g, mpz_t a, mpz_t b)
\end{verbatim}

Simple reimpementation of Eulers Greatest common divisor function.
There are actually two GCD functions in the libary one that annilates the arguments and one that does not.
The one that does not simply creates copies of the arugments and passes them to the ones that do.
Finds the Greatest Common Divisor between a and b.

\subsection{Least Common Multiple }
\begin{verbatim}
lcm( mpz_t, mpz_t, mpz_t)
\end{verbatim}
Simple implementation of the least common Multiple function.
Uses the Greatest Common Divisor to find the similar factors divides them out and then returns the least common multiple.

\subsection{Euler's Extended Algorithm}
\begin{verbatim}
extendedGCD( mpz_t d, mpz_t x, mpz_t y, mpz_t a, mpz_t b )
\end{verbatim}
Simple recursive implementation of Eulers extended Algorithm.
This function, runs in a similar way to the regular GCD function. 
However it runs recursively so that it can go back up the calculation and solve Diphontine Equations.
$$d = ax + by$$

The first three arguments are the return values. The final two are the inputs.


\subsection{Find Modular Inverse}
\begin{verbatim}
bool findModInverse( mpz_t x, mpz_t a, mpz_t n)
\end{verbatim}
This function finds solves the equation $1 = ax \pmod n$. 
It does this  by solving the Diaphontine equation $d = ax + by$ and then checking if $d = 1$. 
If it does not the function returns false.

\subsection{Eulers Totent Function}
\begin{verbatim}
EulerTotent( mpz_t t, mpz_t n)
\end{verbatim}

A GMP implementation of Eulers Totent Function
Returns the number of numbers less than $n$ whose GCD with $n$ is one. 
Also known as relatively prime with n.

Very simply counts up from 1 checking each number for relative primality.


\subsection{Check Generator}
\begin{verbatim}
bool is_Generator( mpz_t m, mpz_t n)
\end{verbatim}
\textbf{ NON-GMP FUNCTION:}
\textbf{Since this function uses the seive of Sunaram to generate primes } 

This function checks if a number $m$ is a primitive root $\mod n$.
While there is no very efficent way to do this, the process can be simplified by using Eulers Totent Function.

The function works by generating the prime factors of $P(\varphi(n)) = p_1,p_2 ... p_n $ and calculating $m^{\frac{\varphi(n)}{p_i}} \pmod n$.
If this exponential is never equal to 1 then the number is a primitive root.

This works using the definition of primitive roots with regard to its multiplicative order which should be equal to the $\phi(n)$. 
The converse of this is also true. Thus we only need to check all non-trivial factors of $\phi(n)$.

\section{Primality}

The concept of Primality is an important one in modern cryptography.
Systems like RSA use very large primes and abstract algebra to create very secure cryptosystems.
As RSA is part of this package a Primality tester is very important.

\subsection{Fermat's Primality Test}
\begin{verbatim}
bool isPrime_Fermat( mpz_t n )
\end{verbatim}

This is a GMP implementaiton of Fermat's Primality Test.
This is a probablistic test to determine primality.
It uses large numbers and Fermat's little theorem to determine if a number is prime.
Fermat's little theorm states $a^{p-1} \equiv 1 \pmod p$, therefore checking a number of $a$ values for this equality can probablisticaly determine primality.

The function begins by initializing GMP's random number generator and checking if the number is divisible by 2.
This check is necessary due to GMP implementation modualar exponentiation requiring odd modular residue classes.
The function then calls the actual function which functions recursively with a counter that instructs how many factors to try.
The default is 5 although this function can be called directly if more are desired.


\subsection{Random Prime}
\begin{verbatim}
random_prime( mpz_t p , int n)
\end{verbatim}

This fucntion returns a prime number between $2^n -1$ and $2^{n+1}-1$.
This is accomplished by generating a number between the bounds and checking if its prime.
If it is not 2 is added to the number and the check is performed again.

\subsection{Sieve of Sundaram}
\begin{verbatim}
Sundaram( size_t n, std::vector<size_t> p)
\end{verbatim}
\textbf{ NON-GMP FUNCTION:}
\textbf{Since this is a Seive and values must be represented in an array there was no reason to make this function GMP} 

The Sieve of Sundaram is an improvment on the Sieve of Eratosthenes as it does not concider even numbers.
The Seive works by making a list of all numbers up to and including $n$ and then crossing out all the numbers of the form
$i + j + 2ji = m$.
Then all remaining numbers are doubled and incremented.

This seive works like Eratosthenes, but since every number is doubled and incremented every number to start is of the form $2n +1$.
We can prove that all $m$ are composite by:
\begin{align*}
q &= 2(i+j+2ij)+1\\
&= 2i +2j + 4ij + 1\\
&= (2i + 1 )( 2j + 1)
\end{align*}
Since all numbers are of the form $2n + 1$ this finds all composite numbers.

The found elements are processed and added to a vector to return.
2 is also added to the vector for completeness sake;

\section{Factorization Algorithms}

Factoring Numbers is a hard problem. 
Much of modern asymmetric key cryptography relies on the intractability of factoring large numbers.
While trial division is a functional solution it has a very long run time.
There are faster ways however, which speed up the factorization by a large margin.
This package contains 3 of them.

\subsection{Pollard Rho}
\begin{verbatim}
bool factor_PollardRho( mpz_t d, mpz_t n)
\end{verbatim}

This function finds a factor $d$ of the number $n$... Sometimes.

The function uses the cyclic nature of applying the output of a polynomial to itself in a modular system.
The fuction performs the operation $x_{k} = g(x_{k-1})$  and $y_{k} \pmod n  = g(g(x_{k-1}))$ repeatedly.
Since this sequence must repeat eventually there will be a time when $x_k \equiv y_k \pmod n$.
This is known as Floyd's cycle finding algorithm.

Every step, the greatest common divisor between $n$ and $|x_k-y_k|$ is computed. 
If the result is ever not relative primality we are done.
One final check is in order however, the algorithm may have found a multiple of $n$, in which case the result will be $n$.
If this occurs the function has failed.
Otherwise the factor is returned.

In the case of failure another function $g(x)$ could be tried, however this implementation only includes the one.


\subsection{Pollard p-1}
\begin{verbatim}
bool factor_Pollardp1( mpz_t g, mpz_t n )
\end{verbatim}
\textbf{ NON-GMP FUNCTION:}
\textbf{ The Seive of Sundaram is used to generate the primes under the smoothness bound. Since Sundaram is non-GMP this is as well.} 
{\bf However since the implemented choice for smoothness bound is the third root of the $n$ this fucntion can handle values up to:
\begin{align*}
(2^{64})^3 &= 6277101735386680763835789423207666416102355444464034512896 \\
&= 6.2 * 10^{57}\\
&= 2^{192} 
\end{align*}
}

Pollard's $p-1$ algorithm is an integer factoring algorithm that uses Fermat's little theorem as it's base.
The algorithm works by first generating a \textbf{smoothness bound} for the number $n$. 
A \textbf{smooth} number is one whose prime factorizaion only contains "small" primes.
The \textbf{smoothness bound} is a guess at about where we want to search for factors of $n$.

Since for coprime $a$ and $p$ $a^{M(p-1)} \equiv 1 mod p$ by Fermat, we know that for this huge number $x = a^{M(P)}$, $gcd(x-1, n)$ will be a factor of $n$.

The main goal of this algorithm is to find $M(p-1)$.
Since we are assuming $n$ to be smooth, we can find the product of all $m_1,m_2...m_n$, under the smoothness bound and take that number to be $M$.
Since there may be multiple of any given factor and modular expoenetiation is comparitvely cheap the actual product for each $m$ is $m_i^k < n$ for the greatest $k$.

This algorithm is good at finding smooth factor but begins to fail at larger ones due to how the smoothenss bound is chosen.
One improvement that could be made is if a factor search fails, to increase the smoothness bound and try again.
This is not done for speed's sake.

\subsubsection{Choosing smoothness bound}
Smoothness bound should not be larger than $\sqrt{n}$ but for this package the smoothness bound is chosen as the third root of n.
This is for additional speed and probability.
By Dixions theorem we know that the probibily of the largest factor of a number being less than $(p-1)^{\\varepsilon}$ is about $\varepsilon^{-\varepsilon}$.
So for the third root, we have around a 60\% chance of finding a factor.

\subsection{Shanks Square Algorithm}
\begin{verbatim}
bool factor_Shank( mpz_t r, mpz_t N)
\end{verbatim}
\textbf{ NON-GMP FUNCTION:}
\textbf{Uses a precompiled list of $k$ values which is limited.} 

Shanks square forms factorizaion algorithm functions by attempting to find $N = x^2 - y^2$, with $x^2 \equiv y^2 \pmod N$.
While it does not guarantee factorization it does provide a good chance that either $(x-y)$ or $(x+y)$ contains a factor of $N$


\section{Tests}

Testing is an important part of code reuse and refactoring.
The cryptomath library has 100\% code coverage for its functions.

These tests can be built by navigating to \textit{Cryptosuite/cryptolib/} and running \textbf{make}.
Run the tests by running \textbf{cryptolib\_unittest}.
This Cryptosuite tests using Google's testing framework.
This framework is not included with this package.
However it can be aquired from \url{https://github.com/google/googletest} or running 
\begin{center}
\textbf{ \$ git clone https://github.com/google/googletest.git} 
\end{center}

from the root directory of the project \textit{Cryptosuite/}.
