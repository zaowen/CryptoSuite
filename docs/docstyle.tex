% !TEX root = DesignDocument.tex



\usepackage[width=6.5in, height=9.2in, top=1.0in, papersize={8.5in,11in}]{geometry}
\usepackage[pdftex]{graphicx}
\usepackage{amsmath}
\usepackage{amsthm}
\usepackage{amssymb}
%\usepackage{txfonts}
\usepackage{textcomp}
\usepackage{amsthm}
\usepackage{algpseudocode}
\usepackage{fancyhdr}
\pagestyle{fancy}
\usepackage{hyperref}
\usepackage{verbatim}


\usepackage{array}
\usepackage{color}
\usepackage{listings}
\usepackage{calc}
\usepackage[utf8]{inputenc}
\usepackage{makeidx}
\usepackage{multicol}
\usepackage{multirow}
\usepackage[table]{xcolor}
\usepackage{tabularx}
\usepackage{framed}
\usepackage{xspace}
\usepackage{etex}
\usepackage{todonotes}


%% Computer Modern Bright Font
%\usepackage{cmbright}
%\usepackage[T1]{fontenc}

%% Sans Serif Modern Font - similar to  Helvetica
\usepackage{lmodern}
\renewcommand*\familydefault{\sfdefault} %% Only if the base font of the document is to be sans serif
\usepackage[T1]{fontenc}


\definecolor{SDColor1}{rgb}{0,0,0}
\definecolor{SDColor2}{rgb}{0,0,0}
\definecolor{SDColor3}{rgb}{0,0,0}
\definecolor{SDColor4}{rgb}{0,0,0}
\definecolor{SDColor5}{rgb}{0,0,0}

%%%  --- Here are some other colors.  Keep it conservative --- %%%

%% Blue font color scheme
%\definecolor{SDColor1}{rgb}{.204,.353,.541}
%\definecolor{SDColor2}{rgb}{.31,.506,.741}
%\definecolor{SDColor3}{rgb}{0.18,0.35,0.59}
%\definecolor{SDColor4}{rgb}{0.44,0.59,0.82}
%\definecolor{SDColor5}{rgb}{0.35,0.35,0.35}
%

% Brown color scheme
% \definecolor{SDColor1}{rgb}{.55,.2,.2}
%\definecolor{SDColor2}{rgb}{.4,.1,.1}
%\definecolor{SDColor3}{rgb}{.5, .15,.15}
%\definecolor{SDColor4}{rgb}{.63,.32,.18}
%\definecolor{SDColor5}{rgb}{.45,.15,.15}
%


%% Custom colors for code listing environment
\definecolor{OliveGreen}{cmyk}{0.64,0,0.95,0.40}
\definecolor{DarkBlue}{cmyk}{0.76,0.76,0,0.20}
\definecolor{DarkRed}{cmyk}{0,1,1,0.45}
\lstset{language=c,frame=ltrb,framesep=5pt,basicstyle=\normalsize,
 keywordstyle=\ttfamily\color{DarkRed},
identifierstyle=\ttfamily\color{DarkBlue}\bfseries,
commentstyle=\color{OliveGreen},
stringstyle=\ttfamily,
showstringspaces=false,tabsize = 3}


\setlength{\oddsidemargin}{0mm} 
\setlength{\evensidemargin}{0mm} 

%% Uncomment if you want "Draft" placed on each page.
%\usepackage{draftwatermark}
%\SetWatermarkLightness{0.975}
%\SetWatermarkScale{1}
%\SetWatermarkText{Draft}

\pagestyle{fancy}
\renewcommand{\chaptermark}[1]{\markboth{#1}{}}
\renewcommand{\sectionmark}[1]{\markright{\thesection\ #1}}
\fancyhf{}
\fancyhead[LE,RO]{\bfseries\thepage}
\fancyhead[LO]{\bfseries\rightmark}
\fancyhead[RE]{\bfseries\leftmark}
%\fancyfoot[LE,RO]{Confidential and Proprietary}
%\renewcommand{\headrulewidth}{0.5pt}
%\renewcommand{\footrulewidth}{0pt}
%\addtolength{\headheight}{0.5pt}
%\setlength{\footskip}{0mm}
%\renewcommand{\footruleskip}{0pt}



\usepackage{titlesec}
\titleformat{\chapter}[display]
{\normalfont\bfseries\color{SDColor3}}    %\normalfont\bfseries\filcenter}
{\LARGE\thechapter}
{1ex}
{\titlerule[2pt]
\vspace{2ex}%
\LARGE}
[\vspace{1ex}%
{\titlerule[2pt]}]

%
%\usepackage{titlesec}
%\titleformat{\chapter}{\normalfont\bfseries\LARGE}
%{\thechapter.}{5pt}{}[{\titlerule[3pt]}]
%
%\titleformat{\section}{\normalfont\bfseries\Large}
%{\thesection.}{5pt}{}[{\titlerule[2pt]}]
%
%\titleformat{\subsection}{\normalfont\bfseries\large}
%{\thesubsection.}{5pt}{}[{\titlerule[1pt]}]
%


%\titleformat*{\section}{\Large\bfseries\sffamily\color{SDColor1}}
%\titleformat*{\subsection}{\large\bfseries\sffamily\color{MSLightBlue}}
%\titleformat*{\section}{\Large\bfseries\color{SDColor3}}
%\titleformat*{\subsection}{\large\bfseries\color{SDColor4}}

%\titleformat*{\section}{\Large\bfseries}
%\titleformat*{\subsection}{\large\bfseries}
%\titleformat*{\subsubsection}{\large\bfseries}

\titleformat*{\section}{\Large\bfseries\color{SDColor1}}  
\titleformat*{\subsection}{\large\bfseries\color{SDColor2}}
\titleformat*{\subsubsection}{\large\bfseries\color{SDColor5}}
\setcounter{secnumdepth}{3}
\renewcommand{\thesubsubsection}{\thesubsection.\alph{subsubsection}}

% Save the original chapter command as stdchapter
\let\stdchapter\chapter

%redefine the backmatter command
\let\stdbackmatter\backmatter
\makeatletter% We need the '@' letter to call if@openright
\renewcommand{\backmatter}{
\stdbackmatter
% need to set the page counter back to 1
\setcounter{page}{1}
%% Redefine the \chapter command for our Back Matter
\renewcommand{\chapter}[1]{
  \if@openright\cleardoublepage\else\clearpage\fi% chapters begin on right page
  \stdchapter{##1}% output standard chapter heading
  \setcounter{section}{0}% restart the section numbering
  \renewcommand{\thepage}{BM-\arabic{page}}% Redefine page numbering format
  \renewcommand{\thesection}{\arabic{section}}% Redefine section number format
}}
\makeatother% Restore the normal behavior of '@'

%redefine the appendix command
\let\stdappendix\appendix
\makeatletter% We need the '@' letter to call if@openright
\renewcommand{\appendix}{
\stdappendix
%% \titleformat{\chapter}[display]
%% {\normalfont\bfseries\color{SDColor3}}    %\normalfont\bfseries\filcenter}
%% {\LARGE Appendix \thechapter}
%% {1ex}
%% {\titlerule[2pt]
%% \vspace{2ex}%
%% \LARGE}
%% [\vspace{1ex}%
%% {\titlerule[2pt]}]
  %%% Since counters are different in the appendix section
  %%% we redefine \chapter to explicitly reset the page number
  %%%  (comment out to see effect)
  \renewcommand{\chapter}[1]{
    \stdchapter{##1}\setcounter{page}{1}
    %%% We also redefine page numbering
    \renewcommand{\thepage}{\Alph{chapter}-\arabic{page}}
  }
}
\makeatother% Restore the normal behavior of '@'


\makeatletter% We need the '@' letter to call if@openright
\newcommand{\agreement}{
  \renewcommand{\chapter}[1]{
    \if@openright\cleardoublepage\else\clearpage\fi% chapters begin on right page
    \pagestyle{plain}% turn off fancy headers
    \setcounter{section}{0}% Reset the section number
    \setcounter{page}{1}% Reset the page number
    \renewcommand{\thepage}{SA-\arabic{page}}% Set format for page numbering
    \renewcommand{\thesection}{\arabic{section}}% Set format for section numbering
    \refstepcounter{chapter}% Add it to the index/toc for on-line viewing
    \addcontentsline{toc}{chapter}{##1}% Add to the table of contents
  }
}
\makeatother% Restore the normal behavior of '@'



%%  If you do some math typesetting, you may want more environment names.
%% Uncomment to see how this works:
%\newtheorem{summary}{Summary:}
%\newtheorem{example}{Example:}



