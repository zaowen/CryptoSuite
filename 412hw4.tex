
\documentclass{article}

\usepackage{fancyhdr}
\usepackage{extramarks}
\usepackage{amsmath}
\usepackage{amsthm}
\usepackage{amsfonts}
\usepackage{tikz}
\usepackage{amssymb}
\usepackage{stackrel}


\usetikzlibrary{automata,positioning}

%
% Basic Document Settings
%

\topmargin=-0.45in
\evensidemargin=0in
\oddsidemargin=0in
\textwidth=6.5in
\textheight=9.0in
\headsep=0.25in

\linespread{1.1}

\pagestyle{fancy}
\lhead{\hmwkAuthorName}
\chead{\hmwkClass\ (\hmwkClassInstructor\ \hmwkClassTime): \hmwkTitle}
\rhead{\firstxmark}
\lfoot{\lastxmark}
\cfoot{\thepage}

\renewcommand\headrulewidth{0.4pt}
\renewcommand\footrulewidth{0.4pt}

\setlength\parindent{0pt}

%
% Create Problem Sections
%

\newcommand{\enterProblemHeader}[1]{
    \nobreak\extramarks{}{Problem \arabic{#1} continued on next page\ldots}\nobreak{}
    \nobreak\extramarks{Problem \arabic{#1} (continued)}{Problem \arabic{#1} continued on next page\ldots}\nobreak{}
}

\newcommand{\exitProblemHeader}[1]{
    \nobreak\extramarks{Problem \arabic{#1} (continued)}{Problem \arabic{#1} continued on next page\ldots}\nobreak{}
    \stepcounter{#1}
    \nobreak\extramarks{Problem \arabic{#1}}{}\nobreak{}
}

\setcounter{secnumdepth}{0}
\newcounter{partCounter}
\newcounter{homeworkProblemCounter}
\setcounter{homeworkProblemCounter}{1}
\nobreak\extramarks{Problem \arabic{homeworkProblemCounter}}{}\nobreak{}

%
% Homework Problem Environment
%
% This environment takes an optional argument. When given, it will adjust the
% problem counter. This is useful for when the problems given for your
% assignment aren't sequential. See the last 3 problems of this template for an
% example.
%
\newenvironment{homeworkProblem}[1][-1]{
    \ifnum#1>0
        \setcounter{homeworkProblemCounter}{#1}
    \fi
    \section{Problem \arabic{homeworkProblemCounter}}
    \setcounter{partCounter}{1}
    \enterProblemHeader{homeworkProblemCounter}
}{
    \exitProblemHeader{homeworkProblemCounter}
}

%
% Homework Details
%   - Title
%   - Due date
%   - Class
%   - Section/Time
%   - Instructor
%   - Author
%

\newcommand{\hmwkTitle}{Homework\ \#2}
\newcommand{\hmwkDueDate}{Nevember 8, 2017}
\newcommand{\hmwkClass}{Cryptography}
\newcommand{\hmwkClassTime}{Section 001}
\newcommand{\hmwkClassInstructor}{Karlsson, Christer H}
\newcommand{\hmwkAuthorName}{\textbf{Zachary Owen}}

%
% Title Page
%

\title{
    \vspace{2in}
    \textmd{\textbf{\hmwkClass:\ \hmwkTitle}}\\
    \normalsize\vspace{0.1in}\small{Due\ on\ \hmwkDueDate\ at 3:00pm}\\
    \vspace{0.1in}\large{\textit{\hmwkClassInstructor\ \hmwkClassTime}}
    \vspace{3in}
}

\author{\hmwkAuthorName}
\date{}

\renewcommand{\part}[1]{\textbf{\large Part \Alph{partCounter}}\stepcounter{partCounter}\\}

%
% Various Helper Commands
%


% Alias for the Solution section header
\newcommand{\solution}{\bigskip \textbf{\large Solution}}

\begin{document}

\maketitle

\pagebreak

\begin{homeworkProblem}
Naive Nelson uses RSA to receive a single ciphertext $c$ , corresponding to the message $m$. His public modulus is $n$ and his public encryption exponent is $e$.  Since he feels guilty that his system was used only once , he agrees to decrypt any ciphertext that someone sends him, as long as it is not $c$ , and return the answer to that person.  Evil Eve sends him the ciphertext $2^e c \pmod n$.  Show how this allows Eve to find $m$.

\solution

Since $c = m^e \pmod n$,

\begin{align*}
2^e c &\pmod n \\
2^e m^e &\pmod n \\
2m^e \rightarrow (2m^e)^d &\pmod n\\
2m &\pmod n \\
\end{align*}

So Evil Eve can obtain the plaintext by multiplying the returned message by $\frac{1}{2} \pmod n$

\end{homeworkProblem}

\pagebreak

\begin{homeworkProblem}

In order to increase security, Bob chooses n and two encryption exponents $e_1$ , $e_2$.  He asks Alice to encrypt her message $m$ to him by first computing $c_1 \equiv m^{e_1} \pmod n$, then encrypting $c_1$ to get $c_2 \equiv c^{e_2}_1 \pmod n$.  Alice then sends $c_2$ to Bob.  Does this double encryption increase security over single encryption?  Why or why not?

\solution

\begin{enumerate}
\item \textbf{If $n$ is prime:} No, since every element in a finite field is a primitive root there must some element $e_0 = e_1e_2$ such that $m^{e_0} = c_2$
\item \textbf{If $n$ is composite} Yes, since $e_0$ does not nessesarily exist.
\end{enumerate} 

\end{homeworkProblem}

\pagebreak

\begin{homeworkProblem}

Show that if $p$ is prime and $a$ and $b$ are integers not divisible by $p$ , with $a^p \equiv b^p \pmod p$, then $a^p \equiv b^p \pmod{p^2}$

\solution

\end{homeworkProblem}

\pagebreak

\begin{homeworkProblem}

Your opponent uses RSA with $n = pq$ and encryption exponent $e$ and encrypts a message $m$.  This yields the ciphertext $c \equiv  m^e \pmod n $.  A spy tells you that, for this message, $m^{12345} \equiv 1 \pmod n$  Describe how to determine $m$.  Note that you do not know $p$, $q$, $\phi(n)$, or the secret decryption exponent $d$.  However, you should find a decryption exponent that works for this particular ciphertext.  Moreover, explain carefully why your decryption works (your explanation must include how the spy’s information is used).

\solution

\end{homeworkProblem}

\pagebreak

\begin{homeworkProblem}

\renewcommand{\labelenumi}{\textbf{\alph{enumi}}}
\begin{enumerate}
\item  Show that the last two decimal digits of a perfect square must be one of the following pairs: $00, el, e4, 25, o6, e9$, where $e$ stands for any even digit and $o$ stands for any odd digit. (Hint: Show that $n^2$ , $(50 + n)^2$, and $(50 - n)^2$ all have the same final decimal digits, and then consider those integers $n$ with $0 \leq n \leq 25$) 
\item Explain how the result of part (a) can be used to speed up Fermat's
factorization method.
\end{enumerate} 

\solution

\end{homeworkProblem}

\end{document}
