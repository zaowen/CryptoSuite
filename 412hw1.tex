\documentclass{article}

\usepackage{fancyhdr}
\usepackage{extramarks}
\usepackage{amsmath}
\usepackage{amsthm}
\usepackage{amsfonts}
\usepackage{tikz}
\usepackage{amssymb}
\usepackage{stackrel}


\usetikzlibrary{automata,positioning}

%
% Basic Document Settings
%

\topmargin=-0.45in
\evensidemargin=0in
\oddsidemargin=0in
\textwidth=6.5in
\textheight=9.0in
\headsep=0.25in

\linespread{1.1}

\pagestyle{fancy}
\lhead{\hmwkAuthorName}
\chead{\hmwkClass\ (\hmwkClassInstructor\ \hmwkClassTime): \hmwkTitle}
\rhead{\firstxmark}
\lfoot{\lastxmark}
\cfoot{\thepage}

\renewcommand\headrulewidth{0.4pt}
\renewcommand\footrulewidth{0.4pt}

\setlength\parindent{0pt}

%
% Create Problem Sections
%

\newcommand{\enterProblemHeader}[1]{
    \nobreak\extramarks{}{Problem \arabic{#1} continued on next page\ldots}\nobreak{}
    \nobreak\extramarks{Problem \arabic{#1} (continued)}{Problem \arabic{#1} continued on next page\ldots}\nobreak{}
}

\newcommand{\exitProblemHeader}[1]{
    \nobreak\extramarks{Problem \arabic{#1} (continued)}{Problem \arabic{#1} continued on next page\ldots}\nobreak{}
    \stepcounter{#1}
    \nobreak\extramarks{Problem \arabic{#1}}{}\nobreak{}
}

\setcounter{secnumdepth}{0}
\newcounter{partCounter}
\newcounter{homeworkProblemCounter}
\setcounter{homeworkProblemCounter}{1}
\nobreak\extramarks{Problem \arabic{homeworkProblemCounter}}{}\nobreak{}

%
% Homework Problem Environment
%
% This environment takes an optional argument. When given, it will adjust the
% problem counter. This is useful for when the problems given for your
% assignment aren't sequential. See the last 3 problems of this template for an
% example.
%
\newenvironment{homeworkProblem}[1][-1]{
    \ifnum#1>0
        \setcounter{homeworkProblemCounter}{#1}
    \fi
    \section{Problem \arabic{homeworkProblemCounter}}
    \setcounter{partCounter}{1}
    \enterProblemHeader{homeworkProblemCounter}
}{
    \exitProblemHeader{homeworkProblemCounter}
}

%
% Homework Details
%   - Title
%   - Due date
%   - Class
%   - Section/Time
%   - Instructor
%   - Author
%

\newcommand{\hmwkTitle}{Homework\ \#1}
\newcommand{\hmwkDueDate}{September 15, 2017}
\newcommand{\hmwkClass}{Cryptography}
\newcommand{\hmwkClassTime}{Section 001}
\newcommand{\hmwkClassInstructor}{Karlsson, Christer H}
\newcommand{\hmwkAuthorName}{\textbf{Zachary Owen}}

%
% Title Page
%

\title{
    \vspace{2in}
    \textmd{\textbf{\hmwkClass:\ \hmwkTitle}}\\
    \normalsize\vspace{0.1in}\small{Due\ on\ \hmwkDueDate\ at 3:00pm}\\
    \vspace{0.1in}\large{\textit{\hmwkClassInstructor\ \hmwkClassTime}}
    \vspace{3in}
}

\author{\hmwkAuthorName}
\date{}

\renewcommand{\part}[1]{\textbf{\large Part \Alph{partCounter}}\stepcounter{partCounter}\\}

%
% Various Helper Commands
%

% Useful for algorithms
\newcommand{\alg}[1]{\textsc{\bfseries \footnotesize #1}}

% For derivatives
\newcommand{\deriv}[1]{\frac{\mathrm{d}}{\mathrm{d}x} (#1)}

% For partial derivatives
\newcommand{\pderiv}[2]{\frac{\partial}{\partial #1} (#2)}

% Integral dx
\newcommand{\dx}{\mathrm{d}x}

% Alias for the Solution section header
\newcommand{\solution}{\textbf{\large Solution}}

\begin{document}

\maketitle

\pagebreak


\begin{homeworkProblem}
Suppose you encrypt a text using an affine cipher, then encrypt the encryption using another affine cipher (both are working mod 26).  Is there any advantage to doing this, rather than using a single affine cipher? Why or why not?\\

\textbf{Solution} \\
	There is no advantage, let the first affine cipher be $e_1(m) = am_i + b$ and the second be $ e_2(m) = cm_i + d $. The double affine function can be represented using functional composition of $e_1$ and $e_2$. 
	\begin{align*}
	 e_1(e_2(m)) &= a(cm_i + d) + b \pmod{26} \\
	 &= acm_i + ad + b \pmod{26} \\
	 &\stackrel{\checkmark}{=} xm_i + y \pmod{26}\\
	 \end{align*}
	 
	 for $x = ac$ and $y = ad+b$. Will the resulting affine cipher be valid? Since non-functional is not an advantage it doesn't matter, but let's prove it anyway. Since the result is $\mod 26$ the value of $y$ does not matter. Since 26 is not prime, $\mathbb{Z}_{26}$ is not a field we don't necessarily know that $gcd(26,ab) = 1$ even if $gcd(a,26) = 1$ and $gcd(b,26) = 1$ or at least I don't. The point is this python program proves $gcd(26,ab) = 1$ 
	\begin{verbatim}
	>>> from fractions import gcd
	>>> prime = list( filter( lambda x: gcd( x, 26 ) == 1, range( 1 , 26 ) )
	>>> prime = [x*y for x in prime for y in prime]
	>>> cross = list( filter( lambda x: gcd( x, 26 ) == 1, prime )
	>>> cross
	[]
	\end{verbatim}
    	

\end{homeworkProblem}

\pagebreak

\begin{homeworkProblem}

Suppose we work mod 27 instead of mod 26 for affine ciphers.  How many keys are possible?  What if we work mod 29? \\

\textbf{Solution} \\

\textit{(From page 15 in the text)} number of affine ciphers in a given base $n$ is equal to $n\cdot f(n)$ where $f(x)$ is the number of integers less than $x$ which are relatively prime with $x$. This short python function shows the number of keys in mod 27 is \textbf{486} and in mod 29 is \textbf{812}
	\begin{verbatim}
	>>> from fractions import gcd
>>> def affinenum( x ):
...     return x * len( list( filter ( lambda y: gcd( x, y ) == 1 , range( 1,x) ) ) 
>>> affinenum(27)
486
>>> affinenum(29)
812
	\end{verbatim}

\end{homeworkProblem}

\pagebreak

\begin{homeworkProblem}

Eve captures Bob’s Hill cipher machine, which uses a 2-by-2 matrix \textbf{M} mod 26.  She tries a chosen plaintext attack.  She finds that the plaintext \textit{ba} encrypts to HC and the plaintext \textit{zz} encrypts to GT.  What is the matrix \textbf{M}. \\


\textbf{Solution} \\
\textit{(From page 37 in the text)} Since the blockcipher is 2-by-2 ciphertext is encrypted my matrix multiplying a message vector $\vec{m}$ by a blockcipher \textbf{M} to obtain a ciphertext vector $\vec{c}$. With some adjustment so that inverse matrices make any sense, we obtain the equation below. And by left multiplying by $m^{-1}$

\begin{align*}
\begin{pmatrix}
1 & 0 \\
25 & 25
\end{pmatrix}
\begin{pmatrix}
M_{1,1} & M_{1,2} \\
M_{2,1} & M_{2,2}
\end{pmatrix} &=
\begin{pmatrix}
7 & 2 \\
6 & 19
\end{pmatrix}\\
\begin{pmatrix}
1 & 0 \\
25 & 25
\end{pmatrix}^{-1}
\begin{pmatrix}
1 & 0 \\
25 & 25
\end{pmatrix}
\begin{pmatrix}
M_{1,1} & M_{1,2} \\
M_{2,1} & M_{2,2}
\end{pmatrix} &=
\begin{pmatrix}
1 & 0 \\
25 & 25
\end{pmatrix}^{-1}
\begin{pmatrix}
7 & 2 \\
6 & 19
\end{pmatrix}\\
\begin{pmatrix}
M_{1,1} & M_{1,2} \\
M_{2,1} & M_{2,2}
\end{pmatrix} 
&\stackrel{\checkmark}{=}
\begin{pmatrix}
a & b \\
c & d
\end{pmatrix}
\end{align*}

Notice when we multiply $c$ by $m^{-1}$ we are left with the blockcipher. With $m^{-1} = \begin{pmatrix}
1 & 0 \\
25 & 25
\end{pmatrix}$ then 

\begin{align*}
\begin{pmatrix}
1 & 0 \\
25 & 25
\end{pmatrix}
\begin{pmatrix}
1 & 0 \\
25 & 25
\end{pmatrix}
\begin{pmatrix}
M_{1,1} & M_{1,2} \\
M_{2,1} & M_{2,2}
\end{pmatrix} &=
\begin{pmatrix}
1 & 0 \\
25 & 25
\end{pmatrix}
\begin{pmatrix}
7 & 2 \\
6 & 19
\end{pmatrix}\\
\begin{pmatrix}
M_{1,1} & M_{1,2} \\
M_{2,1} & M_{2,2}
\end{pmatrix} 
&\stackrel{\checkmark}{=}
\begin{pmatrix}
7 & 2 \\
13 & 5
\end{pmatrix}
\end{align*}

\end{homeworkProblem}

\pagebreak

\begin{homeworkProblem}

A sequence generated by a length three recurrence starts 001110.  Find the next four elements of the sequence. \\
\textbf{Solution} \\

\begin{align*}
1 &= c_0 \cdot 0 + c_1 \cdot 0 + c_2 \cdot 1 &(n = 1) \\
1 &= c_0 \cdot 0 + c_1 \cdot 1 + c_2 \cdot 1 &(n = 2) \\
0 &= c_0 \cdot 1 + c_1 \cdot 1 + c_2 \cdot 1 &(n = 3)
\end{align*}

\begin{align*}
\begin{pmatrix}
0& 0 & 1 \\
0& 1 & 1 \\
1& 1 & 1 \\
\end{pmatrix}
\begin{pmatrix}
c_0 \\
c_1 \\
c_2 \\
\end{pmatrix} &\equiv
\begin{pmatrix}
1\\
1 \\
0 \\
\end{pmatrix} \\
\begin{pmatrix}
0& 0 & 1 \\
0& 1 & 1 \\
1& 1 & 1 \\
\end{pmatrix}^{-1}
\begin{pmatrix}
0& 0 & 1 \\
0& 1 & 1 \\
1& 1 & 1 \\
\end{pmatrix}
\begin{pmatrix}
c_0 \\
c_1 \\
c_2 \\
\end{pmatrix} &\equiv
\begin{pmatrix}
0& 0 & 1 \\
0& 1 & 1 \\
1& 1 & 1 \\
\end{pmatrix}^{-1}
\begin{pmatrix}
1\\
1 \\
0 \\
\end{pmatrix}
\\
\begin{pmatrix}
0& 1 & 1 \\
1& 1 & 0 \\
1& 0 & 0 \\
\end{pmatrix}
\begin{pmatrix}
0& 0 & 1 \\
0& 1 & 1 \\
1& 1 & 1 \\
\end{pmatrix}
\begin{pmatrix}
c_0 \\
c_1 \\
c_2 \\
\end{pmatrix} &\equiv
\begin{pmatrix}
0& 1 & 1 \\
1& 1 & 0 \\
1& 0 & 0 \\
\end{pmatrix}
\begin{pmatrix}
1\\
1 \\
0 \\
\end{pmatrix}
\\
\begin{pmatrix}
c_0 \\
c_1 \\
c_2 \\
\end{pmatrix} 
&\stackrel{\checkmark}{\equiv}
\begin{pmatrix}
1 \\
0 \\
1 \\
\end{pmatrix} 
\end{align*}

Since this sequence is generated by a  recurrence relation of order three, it's characteristic equation is $m_n = c_0 \cdot m_{n-3} + c_1 \cdot m_{n-2} + c_2 \cdot m_{n-1}$ and specifically $m_n = m_{n-3} + m_{n-1} \pmod 2$. This means the next four bits are \textit{001110\textbf{1001}}


\end{homeworkProblem}

\pagebreak

\begin{homeworkProblem}

Experiment with the affine cipher $y = m x + n \pmod{26}$ for values of m $>$ 26.  In particular, determine whether or not these encryptions are the same as ones obtained with m $<$ 26.\\

\textbf{Solution} \\
Let integer $k$ in a system modulo $26$ be represented by the equation $k = a26 + b$.Two affine ciphers are identical if  $d y + f \equiv m x + n \pmod{26} \forall yx \in \mathbb{Z}$ Let $y$ be an integer greater than 26\\
\begin{align*}
a x + n &\equiv m x + n \pmod{26} \\
( d 26 + z )x + n &\equiv m x + n \pmod{26} \\
xd26 + xz + n &\equiv m x + n \pmod{26}\\
xz + n &\equiv m x + n \pmod{26}
\end{align*}
Therefore affine cipher encryptions with $m>26$ are identical to ones $m<26$ given that $z = m$ which is true when $a \equiv m \pmod{26}$ 



\end{homeworkProblem}

\pagebreak

\begin{homeworkProblem}

The following sequence was generated by a linear feedback shift register.  Determine the recurrence that generated it. \\

[1, 0, 1, 0, 0, 1, 1, 0, 1, 1,
 0, 0, 0, 1, 0, 0, 1, 0, 0, 0, 
 0, 1, 1, 1, 0, 0, 0, 0, 0, 1,
 0, 1, 1, 1, 1, 1, 1, 0, 0, 1, 
 0, 1, 0, 1, 0, 0, 0, 1, 1, 0, 
 0, 1, 1, 1, 1, 0, 1, 1, 1, 0, 
 1, 0, 1, 
 1, 0, 1, 0, 0, 1, 1, 0, 1, 1,
 0, 0, 0, 1, 0, 0, 1, 0, 0, 0,
 0, 1, 1, 1, 0, 0, 0]\\

\textbf{Solution} \\

First, notice that the sequence begins to repeat at bit 63 
\begin{align*}
1, 0, &1, 0, 0, 1, 1, 0, 1, 1, 0, 0, 0, 1, 0, 0, 1, 0,\cdots \\ \cdots 0, 1, 1, 0,&1, 0, 0, 1, 1, 0, 1, 1, 
\end{align*}
Since the recurrence relationship is based on the previous state, if the sequence does not begin to repeat after $2^n$ then the order of the recurrence relationship cannot be of order less than $n$. 
Since $63 > 2 ^ n$, $n$ must be at least 6. Let's assume it is 6...
\begin{align*}
1 &= c_0 \cdot 1 + c_1 \cdot 0 + c_2 \cdot 1 + c_3 \cdot 0 + c_4 \cdot 0 + c_5 \cdot 1 &(n = 1) \\
0 &= c_0 \cdot 0 + c_1 \cdot 1 + c_2 \cdot 0 + c_3 \cdot 0 + c_4 \cdot 1 + c_5 \cdot 1 &(n = 2) \\
1 &= c_0 \cdot 1 + c_1 \cdot 0 + c_2 \cdot 0 + c_3 \cdot 1 + c_4 \cdot 1 + c_5 \cdot 0 &(n = 3) \\
1 &= c_0 \cdot 0 + c_1 \cdot 0 + c_2 \cdot 1 + c_3 \cdot 1 + c_4 \cdot 0 + c_5 \cdot 1 &(n = 4) \\
0 &= c_0 \cdot 0 + c_1 \cdot 1 + c_2 \cdot 1 + c_3 \cdot 0 + c_4 \cdot 1 + c_5 \cdot 1 &(n = 5) \\
0 &= c_0 \cdot 1 + c_1 \cdot 1 + c_2 \cdot 0 + c_3 \cdot 1 + c_4 \cdot 1 + c_5 \cdot 0 &(n = 6) \\
\end{align*}

\begin{align*}
\begin{pmatrix}
1 &0 &1 &0 &0 &1 \\
0 &1 &0 &0 &1 &1 \\
1 &0 &0 &1 &1 &0 \\
0 &0 &1 &1 &0 &1 \\ 
0 &1 &1 &0 &1 &1 \\
1 &1 &0 &1 &1 &0 \\
\end{pmatrix}
\begin{pmatrix}
c_0 \\
c_1 \\
c_2 \\
c_3 \\
c_4 \\
c_5 \\
\end{pmatrix} &\equiv
\begin{pmatrix}
1 \\
0 \\
1 \\
1 \\
0 \\
0 \\
\end{pmatrix} 
\\
\begin{pmatrix}
1 &0 &1 &0 &0 &1 \\
0 &1 &0 &0 &1 &1 \\
1 &0 &0 &1 &1 &0 \\
0 &0 &1 &1 &0 &1 \\ 
0 &1 &1 &0 &1 &1 \\
1 &1 &0 &1 &1 &0 \\
\end{pmatrix}^{-1}
\begin{pmatrix}
1 &0 &1 &0 &0 &1 \\
0 &1 &0 &0 &1 &1 \\
1 &0 &0 &1 &1 &0 \\
0 &0 &1 &1 &0 &1 \\ 
0 &1 &1 &0 &1 &1 \\
1 &1 &0 &1 &1 &0 \\
\end{pmatrix}
\begin{pmatrix}
c_0 \\
c_1 \\
c_2 \\
c_3 \\
c_4 \\
c_5 \\
\end{pmatrix} &\equiv
\begin{pmatrix}
1 &0 &1 &0 &0 &1 \\
0 &1 &0 &0 &1 &1 \\
1 &0 &0 &1 &1 &0 \\
0 &0 &1 &1 &0 &1 \\ 
0 &1 &1 &0 &1 &1 \\
1 &1 &0 &1 &1 &0 \\
\end{pmatrix}^{-1}
\begin{pmatrix}
1 \\
0 \\
1 \\
1 \\
0 \\
0 \\
\end{pmatrix}
\\
\begin{pmatrix}
c_0 \\
c_1 \\
c_2 \\
c_3 \\
c_4 \\
c_5 \\
\end{pmatrix} &\equiv
\begin{pmatrix}
0 &1 &0 &1 &1 &0 \\
1 &0 &1 &1 &0 &0 \\
0 &1 &1 &0 &0 &1 \\
1 &1 &0 &0 &1 &0 \\ 
1 &0 &0 &1 &0 &0 \\
0 &0 &1 &0 &0 &1 \\
\end{pmatrix}
\begin{pmatrix}
1 \\
0 \\
1 \\
1 \\
0 \\
0 \\
\end{pmatrix}\\
\begin{pmatrix}
c_0 \\
c_1 \\
c_2 \\
c_3 \\
c_4 \\
c_5 \\
\end{pmatrix} &\equiv
\begin{pmatrix}
1 \\
1 \\
1 \\
1 \\
0 \\
1 \\
\end{pmatrix} 
\end{align*}
Meaning the characteristic equation for this reccurance relationship is $m_n = c_0 \cdot m_{n-6} + c_1 \cdot m_{n-5} +c_2 \cdot m_{n-4} +c_3 \cdot m_{n-3} + c_6 \cdot m_{n-1}$ 





\end{homeworkProblem} 

\pagebreak

\begin{homeworkProblem}

You have caught the following ciphertext, and you believe it was encrypted by an affine cipher: $$edsgickxhuklzveqzvkxwkzukcvuh$$ You also believe that the first two letters of the plaintext are \textit{if}.  Decrypt.\\

\textbf{Solution} \\
\textit{(From page 15 in the text)} Since it is given the first two letters are \textit{if} and by the nature of affine ciphers we have these two equations.
$8\alpha+\beta = 4$ and $5\alpha+\beta=3 \pmod{26}$. Subtracting the two gives the equation $3\alpha = 1 \pmod{26}$. And since $gcd(3,26)=1$, this equation has the unique solution $\alpha = 9$, which in turn gives the solution $\beta = 10$. From here the process is simply mechanical.
\center
\begin{tabular}{ l | l | l | l | l | l | l | l | l | l | l | l | l | l | l | l | l | l | l | l | l | l | l | l | l | l | l | l | l  }
e &d &s &g &i &c &k &x &h &u &k &l &z &v &e &q &z &v &k &x &w &k &z &u &k &c &v &u &h \\
\hline
i &f &y &o &u &c &a &n &r &e &a &d &t &h &i &s &t &h &a &n &k &a &t &e &a &c &h &e &r
\end{tabular}

ifyoucanreadthisthankateacher


\end{homeworkProblem}
	
\end{document}