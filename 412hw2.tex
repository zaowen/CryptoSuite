\documentclass{article}

\usepackage{fancyhdr}
\usepackage{extramarks}
\usepackage{amsmath}
\usepackage{amsthm}
\usepackage{amsfonts}
\usepackage{tikz}
\usepackage{amssymb}
\usepackage{stackrel}


\usetikzlibrary{automata,positioning}

%
% Basic Document Settings
%

\topmargin=-0.45in
\evensidemargin=0in
\oddsidemargin=0in
\textwidth=6.5in
\textheight=9.0in
\headsep=0.25in

\linespread{1.1}

\pagestyle{fancy}
\lhead{\hmwkAuthorName}
\chead{\hmwkClass\ (\hmwkClassInstructor\ \hmwkClassTime): \hmwkTitle}
\rhead{\firstxmark}
\lfoot{\lastxmark}
\cfoot{\thepage}

\renewcommand\headrulewidth{0.4pt}
\renewcommand\footrulewidth{0.4pt}

\setlength\parindent{0pt}

%
% Create Problem Sections
%

\newcommand{\enterProblemHeader}[1]{
    \nobreak\extramarks{}{Problem \arabic{#1} continued on next page\ldots}\nobreak{}
    \nobreak\extramarks{Problem \arabic{#1} (continued)}{Problem \arabic{#1} continued on next page\ldots}\nobreak{}
}

\newcommand{\exitProblemHeader}[1]{
    \nobreak\extramarks{Problem \arabic{#1} (continued)}{Problem \arabic{#1} continued on next page\ldots}\nobreak{}
    \stepcounter{#1}
    \nobreak\extramarks{Problem \arabic{#1}}{}\nobreak{}
}

\setcounter{secnumdepth}{0}
\newcounter{partCounter}
\newcounter{homeworkProblemCounter}
\setcounter{homeworkProblemCounter}{1}
\nobreak\extramarks{Problem \arabic{homeworkProblemCounter}}{}\nobreak{}

%
% Homework Problem Environment
%
% This environment takes an optional argument. When given, it will adjust the
% problem counter. This is useful for when the problems given for your
% assignment aren't sequential. See the last 3 problems of this template for an
% example.
%
\newenvironment{homeworkProblem}[1][-1]{
    \ifnum#1>0
        \setcounter{homeworkProblemCounter}{#1}
    \fi
    \section{Problem \arabic{homeworkProblemCounter}}
    \setcounter{partCounter}{1}
    \enterProblemHeader{homeworkProblemCounter}
}{
    \exitProblemHeader{homeworkProblemCounter}
}

%
% Homework Details
%   - Title
%   - Due date
%   - Class
%   - Section/Time
%   - Instructor
%   - Author
%

\newcommand{\hmwkTitle}{Homework\ \#2}
\newcommand{\hmwkDueDate}{October 4, 2017}
\newcommand{\hmwkClass}{Cryptography}
\newcommand{\hmwkClassTime}{Section 001}
\newcommand{\hmwkClassInstructor}{Karlsson, Christer H}
\newcommand{\hmwkAuthorName}{\textbf{Zachary Owen}}

%
% Title Page
%

\title{
    \vspace{2in}
    \textmd{\textbf{\hmwkClass:\ \hmwkTitle}}\\
    \normalsize\vspace{0.1in}\small{Due\ on\ \hmwkDueDate\ at 3:00pm}\\
    \vspace{0.1in}\large{\textit{\hmwkClassInstructor\ \hmwkClassTime}}
    \vspace{3in}
}

\author{\hmwkAuthorName}
\date{}

\renewcommand{\part}[1]{\textbf{\large Part \Alph{partCounter}}\stepcounter{partCounter}\\}

%
% Various Helper Commands
%

% Alias for the Solution section header
\newcommand{\solution}{\newline \textbf{\large Solution} \newline }

\begin{document}

\maketitle

\pagebreak


\begin{homeworkProblem}

Let p be prime.  Show that $a^p \equiv a \pmod p$ for all $a$.
\newline
\solution

\end{homeworkProblem}

\pagebreak

\begin{homeworkProblem}
Let $p \geq 3$ be prime. Show that the only solutions to $x^2 \equiv 1 \pmod p$  are $x \equiv \pm 1 \pmod p$. 
\newline
\solution \\

\end{homeworkProblem}

\pagebreak

\begin{homeworkProblem}
\renewcommand{\labelenumi}{\textbf{\alph{enumi}}}
\begin{itemize}
\item Find all four solutions to $x^2 \equiv 133 \pmod{143}$.
\item Find all solutions to $x^2 \equiv 77 \pmod {143}$  Hint: There are only two solutions, as $ gcd(77, 143) \neq 1$ 
 \end{itemize}\
\solution

\end{homeworkProblem}

\pagebreak

\begin{homeworkProblem}
Find solutions to:
\renewcommand{\labelenumi}{\textbf{\alph{enumi}}}
\begin{enumerate}
 \item $3x - 15y = 2$
 \item $3x - 14y = 2$
\end{enumerate}\
\solution

\end{homeworkProblem}

\pagebreak

\begin{homeworkProblem}
Prove the following theorem \\
	\textbf{Theorem:} Let $p$ be a positive prime and $g$ be a primitive root modulo $p$.
	\begin{itemize}
	\item[] 1. Let $n$ be an integer, then \\
	$g^n \equiv 1 \pmod p$ if and only if $n \equiv 0 \pmod{p - 1}$. \\
	2. If $j$ and $k$ are integers, then \\
	 $g^j \equiv g^k (mod p)$ if and only if $j \equiv k \pmod{p - 1}$.
	 \end{itemize}
	 You are welcome to read the proof on page 84, but your solution must be written in your own words. \newline
\solution 

\end{homeworkProblem}

\pagebreak

\begin{homeworkProblem}
\renewcommand{\labelenumi}{\alph{enumi}}
\begin{enumerate}
\item Let $p$ be a positive prime.  Define a \textit{primitive root modulo} $p$
\item Identify all primitive roots modulo 11.  Is your solution consistent with the claim that there are $\varphi(\varphi(p))$ primitive roots \textit{modulo} p?
\item We stated the \textit{Primitive Root Theorem}:  If $p$ is prime, then there is at least one primitive root modulo $p$.  Show that this result does not hold for composite $n$: if $n$ is composite, then there may not be unit that is a multiplicative generator (ie, primitive root) of the set of units modulo $n$.  Hint: Check modulo 8. 
\end{enumerate}\
\solution

\end{homeworkProblem}

\pagebreak

\begin{homeworkProblem}
Question 27 in the book.
\solution

\end{homeworkProblem}
	
\end{document}