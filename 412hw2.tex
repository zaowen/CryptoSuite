\documentclass{article}

\usepackage{fancyhdr}
\usepackage{extramarks}
\usepackage{amsmath}
\usepackage{amsthm}
\usepackage{amsfonts}
\usepackage{tikz}
\usepackage{amssymb}
\usepackage{stackrel}


\usetikzlibrary{automata,positioning}

%
% Basic Document Settings
%

\topmargin=-0.45in
\evensidemargin=0in
\oddsidemargin=0in
\textwidth=6.5in
\textheight=9.0in
\headsep=0.25in

\linespread{1.1}

\pagestyle{fancy}
\lhead{\hmwkAuthorName}
\chead{\hmwkClass\ (\hmwkClassInstructor\ \hmwkClassTime): \hmwkTitle}
\rhead{\firstxmark}
\lfoot{\lastxmark}
\cfoot{\thepage}

\renewcommand\headrulewidth{0.4pt}
\renewcommand\footrulewidth{0.4pt}

\setlength\parindent{0pt}

%
% Create Problem Sections
%

\newcommand{\enterProblemHeader}[1]{
    \nobreak\extramarks{}{Problem \arabic{#1} continued on next page\ldots}\nobreak{}
    \nobreak\extramarks{Problem \arabic{#1} (continued)}{Problem \arabic{#1} continued on next page\ldots}\nobreak{}
}

\newcommand{\exitProblemHeader}[1]{
    \nobreak\extramarks{Problem \arabic{#1} (continued)}{Problem \arabic{#1} continued on next page\ldots}\nobreak{}
    \stepcounter{#1}
    \nobreak\extramarks{Problem \arabic{#1}}{}\nobreak{}
}

\setcounter{secnumdepth}{0}
\newcounter{partCounter}
\newcounter{homeworkProblemCounter}
\setcounter{homeworkProblemCounter}{1}
\nobreak\extramarks{Problem \arabic{homeworkProblemCounter}}{}\nobreak{}

%
% Homework Problem Environment
%
% This environment takes an optional argument. When given, it will adjust the
% problem counter. This is useful for when the problems given for your
% assignment aren't sequential. See the last 3 problems of this template for an
% example.
%
\newenvironment{homeworkProblem}[1][-1]{
    \ifnum#1>0
        \setcounter{homeworkProblemCounter}{#1}
    \fi
    \section{Problem \arabic{homeworkProblemCounter}}
    \setcounter{partCounter}{1}
    \enterProblemHeader{homeworkProblemCounter}
}{
    \exitProblemHeader{homeworkProblemCounter}
}

%
% Homework Details
%   - Title
%   - Due date
%   - Class
%   - Section/Time
%   - Instructor
%   - Author
%

\newcommand{\hmwkTitle}{Homework\ \#2}
\newcommand{\hmwkDueDate}{October 4, 2017}
\newcommand{\hmwkClass}{Cryptography}
\newcommand{\hmwkClassTime}{Section 001}
\newcommand{\hmwkClassInstructor}{Karlsson, Christer H}
\newcommand{\hmwkAuthorName}{\textbf{Zachary Owen}}

%
% Title Page
%

\title{
    \vspace{2in}
    \textmd{\textbf{\hmwkClass:\ \hmwkTitle}}\\
    \normalsize\vspace{0.1in}\small{Due\ on\ \hmwkDueDate\ at 3:00pm}\\
    \vspace{0.1in}\large{\textit{\hmwkClassInstructor\ \hmwkClassTime}}
    \vspace{3in}
}

\author{\hmwkAuthorName}
\date{}

\renewcommand{\part}[1]{\textbf{\large Part \Alph{partCounter}}\stepcounter{partCounter}\\}

%
% Various Helper Commands
%

% Useful for algorithms
\newcommand{\alg}[1]{\textsc{\bfseries \footnotesize #1}}

% For derivatives
\newcommand{\deriv}[1]{\frac{\mathrm{d}}{\mathrm{d}x} (#1)}

% For partial derivatives
\newcommand{\pderiv}[2]{\frac{\partial}{\partial #1} (#2)}

% Integral dx
\newcommand{\dx}{\mathrm{d}x}

% Alias for the Solution section header
\newcommand{\solution}{\textbf{\large Solution}}

\begin{document}

\maketitle

\pagebreak


\begin{homeworkProblem}

Let p be prime.  Show that $a^p \equiv a \pmod p$ for all $a$. \\\\
\solution \\

\end{homeworkProblem}

\pagebreak

\begin{homeworkProblem}
2. Let p ≥ 3 be prime. Show that the only solutions to x2 ≡ 1 (mod p)  are x ≡ ±1 (mod p). \\\\
\solution \\

\end{homeworkProblem}

\pagebreak

\begin{homeworkProblem}
3.     a         b Find all four solutions to x2 ≡ 133 (mod 143). Find all solutions to x2 ≡ 77 (mod 143)  Hint: There are only two solutions, as gcd(77, 143) ≠ 1 
\solution \\

\end{homeworkProblem}

\pagebreak

\begin{homeworkProblem}
4.         a         b Find solutions to: 3x - 15y = 2 3x - 14y = 2 \\\\
\solution \\

\end{homeworkProblem}

\pagebreak

\begin{homeworkProblem}
5. Prove the following theorem Theorem: Let p be a positive prime and g be a primitive root modulo p. 1. Let n be an integer, then gn ≡ 1 (mod p) if and only if n ≡ 0 (mod p - 1). 2. If j and k are integers, then gj ≡ gk (mod p) if and only if j ≡ k (mod p - 1). You are welcome to read the proof on page 84, but your solution must be written in your own words. \\\\
\solution \\

\end{homeworkProblem}

\pagebreak

\begin{homeworkProblem}
6.      a         b         c Let p be a positive prime.  Define a primitive root modulo pIdentify all primitive roots modulo 11.  Is your solution consistent with the claim that there are φ(φ(p)) primitive roots modulo p? We stated the Primitive Root Theorem:  If p is prime, then there is at least one primitive root modulo p.  Show that this result does not hold for composite n: if n is composite, then there may not be unit that is a multiplicative generator (ie, primitive root) of the set of units modulo n.  Hint: Check modulo 8. \\\\
\textbf{Solution} \\

\end{homeworkProblem}

\pagebreak

\begin{homeworkProblem}
7. Question 27 in the book.

\textbf{Solution} \\

\end{homeworkProblem}
	
\end{document}