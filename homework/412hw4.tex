
\documentclass{article}

\usepackage{fancyhdr}
\usepackage{extramarks}
\usepackage{amsmath}
\usepackage{amsthm}
\usepackage{amsfonts}
\usepackage{tikz}
\usepackage{amssymb}
\usepackage{stackrel}


\usetikzlibrary{automata,positioning}

%
% Basic Document Settings
%

\topmargin=-0.45in
\evensidemargin=0in
\oddsidemargin=0in
\textwidth=6.5in
\textheight=9.0in
\headsep=0.25in

\linespread{1.1}

\pagestyle{fancy}
\lhead{\hmwkAuthorName}
\chead{\hmwkClass\ (\hmwkClassInstructor\ \hmwkClassTime): \hmwkTitle}
\rhead{\firstxmark}
\lfoot{\lastxmark}
\cfoot{\thepage}

\renewcommand\headrulewidth{0.4pt}
\renewcommand\footrulewidth{0.4pt}

\setlength\parindent{0pt}

%
% Create Problem Sections
%

\newcommand{\enterProblemHeader}[1]{
    \nobreak\extramarks{}{Problem \arabic{#1} continued on next page\ldots}\nobreak{}
    \nobreak\extramarks{Problem \arabic{#1} (continued)}{Problem \arabic{#1} continued on next page\ldots}\nobreak{}
}

\newcommand{\exitProblemHeader}[1]{
    \nobreak\extramarks{Problem \arabic{#1} (continued)}{Problem \arabic{#1} continued on next page\ldots}\nobreak{}
    \stepcounter{#1}
    \nobreak\extramarks{Problem \arabic{#1}}{}\nobreak{}
}

\setcounter{secnumdepth}{0}
\newcounter{partCounter}
\newcounter{homeworkProblemCounter}
\setcounter{homeworkProblemCounter}{1}
\nobreak\extramarks{Problem \arabic{homeworkProblemCounter}}{}\nobreak{}

%
% Homework Problem Environment
%
% This environment takes an optional argument. When given, it will adjust the
% problem counter. This is useful for when the problems given for your
% assignment aren't sequential. See the last 3 problems of this template for an
% example.
%
\newenvironment{homeworkProblem}[1][-1]{
    \ifnum#1>0
        \setcounter{homeworkProblemCounter}{#1}
    \fi
    \section{Problem \arabic{homeworkProblemCounter}}
    \setcounter{partCounter}{1}
    \enterProblemHeader{homeworkProblemCounter}
}{
    \exitProblemHeader{homeworkProblemCounter}
}

%
% Homework Details
%   - Title
%   - Due date
%   - Class
%   - Section/Time
%   - Instructor
%   - Author
%

\newcommand{\hmwkTitle}{Homework\ \#2}
\newcommand{\hmwkDueDate}{Nevember 8, 2017}
\newcommand{\hmwkClass}{Cryptography}
\newcommand{\hmwkClassTime}{Section 001}
\newcommand{\hmwkClassInstructor}{Karlsson, Christer H}
\newcommand{\hmwkAuthorName}{\textbf{Zachary Owen}}

%
% Title Page
%

\title{
    \vspace{2in}
    \textmd{\textbf{\hmwkClass:\ \hmwkTitle}}\\
    \normalsize\vspace{0.1in}\small{Due\ on\ \hmwkDueDate\ at 3:00pm}\\
    \vspace{0.1in}\large{\textit{\hmwkClassInstructor\ \hmwkClassTime}}
    \vspace{3in}
}

\author{\hmwkAuthorName}
\date{}

\renewcommand{\part}[1]{\textbf{\large Part \Alph{partCounter}}\stepcounter{partCounter}\\}

%
% Various Helper Commands
%


% Alias for the Solution section header
\newcommand{\solution}{\bigskip \textbf{\large Solution}}

\begin{document}

\maketitle

\pagebreak

\begin{homeworkProblem}
Naive Nelson uses RSA to receive a single ciphertext $c$ , corresponding to the message $m$. His public modulus is $n$ and his public encryption exponent is $e$.  Since he feels guilty that his system was used only once , he agrees to decrypt any ciphertext that someone sends him, as long as it is not $c$ , and return the answer to that person.  Evil Eve sends him the ciphertext $2^e c \pmod n$.  Show how this allows Eve to find $m$.

\solution

Since $c = m^e \pmod n$,

\begin{align*}
2^e c &\pmod n \\
2^e m^e &\pmod n \\
2m^e \rightarrow (2m^e)^d &\pmod n\\
2m &\pmod n \\
\end{align*}

So Evil Eve can obtain the plaintext by multiplying the returned message by $\frac{1}{2} \pmod n$

\end{homeworkProblem}

\pagebreak

\begin{homeworkProblem}

In order to increase security, Bob chooses n and two encryption exponents $e_1$ , $e_2$.  He asks Alice to encrypt her message $m$ to him by first computing $c_1 \equiv m^{e_1} \pmod n$, then encrypting $c_1$ to get $c_2 \equiv c^{e_2}_1 \pmod n$.  Alice then sends $c_2$ to Bob.  Does this double encryption increase security over single encryption?  Why or why not?

\solution

\begin{enumerate}
\item \textbf{If $n$ is prime:} No, since every element in a finite field is a primitive root there must some element $e_0 = e_1e_2$ such that $m^{e_0} = c_2$
\item \textbf{If $n$ is composite} Yes, since $e_0$ does not necessarily exist.
\end{enumerate} 

\end{homeworkProblem}

\pagebreak

\begin{homeworkProblem}

Show that if $p$ is prime and $a$ and $b$ are integers not divisible by $p$ , with $a^p \equiv b^p \pmod p$, then $a^p \equiv b^p \pmod{p^2}$

\solution

By Fermat's Little Theorem we know that $a^p \equiv a \pmod{p}$. We also know $a = pk + a_0$ for some $k$ by the definition of division. This applies without loss of generality to $b$. Meaning $a^p = (pk + a_0)^p$.

We can then expand $(pk+a_0)^p \pmod p^2$ by observing that $p^n \equiv 0 \pmod{p^2}$ for $n \geq 2$. Thus:
\begin{align*}
a^p &= (pk+a_0)^p \pmod p^2 \\
&= \binom{p}{p-1}pk + a_0 \pmod p^2 \\
&= ppk + a_0 \pmod p^2 \\
&= a_0 \pmod p^2 \\
\end{align*}

Therefore, $a_0 \equiv b_0 \pmod{p^2}$

\end{homeworkProblem}

\pagebreak

\begin{homeworkProblem}

Your opponent uses RSA with $n = pq$ and encryption exponent $e$ and encrypts a message $m$.  This yields the ciphertext $c \equiv  m^e \pmod n $.  A spy tells you that, for this message, $m^{12345} \equiv 1 \pmod n$  Describe how to determine $m$.  Note that you do not know $p$, $q$, $\phi(n)$, or the secret decryption exponent $d$.  However, you should find a decryption exponent that works for this particular ciphertext.  Moreover, explain carefully why your decryption works (your explanation must include how the spy’s information is used).

\solution

There are two possiblities, either $gcd( 12345, e) = 1$ or it doesn't. If it does we can use the Extended Euclidian Algorithm to calculate the inverse of $e \pmod {12345} = d_0$.

$$ed_0 = 12345a + 1$$

Then we can raise the ciphertext to $d_0$ and the rest is algebra

\begin{align*}
c^{d_0}  &\pmod n \\
(m^e)^{d_0}  &\pmod n \\
m^{ed_0}  &\pmod n \\
m^{12345a + 1}  &\pmod n \\
m^{12345a }m  &\pmod n \\
1\cdot m  &\pmod n \\
\end{align*}

In the case that $gcd(12345, e) \neq 1$  and the final step looks like 
$1\cdot m  \pmod n $
it is possible to continuously multiply the result by $m$ until the messange makes sense. or if the gcd is known simply multiply it by $m$ raised to the diffrence.

\end{homeworkProblem}

\pagebreak

\begin{homeworkProblem}

\renewcommand{\labelenumi}{\textbf{\alph{enumi}}}
\begin{enumerate}
\item  Show that the last two decimal digits of a perfect square must be one of the following pairs: $00, el, e4, 25, o6, e9$, where $e$ stands for any even digit and $o$ stands for any odd digit. (Hint: Show that $n^2$ , $(50 + n)^2$, and $(50 - n)^2$ all have the same final decimal digits, and then consider those integers $n$ with $0 \leq n \leq 25$) 
\item Explain how the result of part (a) can be used to speed up Fermat's
factorization method.
\end{enumerate} 

\solution

\begin{enumerate}
\item First, note that since we only care about the final 2 decimal digits we can work in $\mod{100}$  

Second, observe that the expansions of $(50 + n)^2$, and $(50 - n)^2$ are equivalent to $n^2$.
\begin{align*}
(50 + n)^2 &= 2500 + 100n + n^2 & \equiv n^2 \pmod{100} \\
(50 - n)^2 &= 2500 - 100n + n^2 & \equiv n^2 \pmod{100} \\
\end{align*}
and note that the functions $(50 + n)^2$, $(50 - n)^2$, and $n^2$ represent all of $\mathbb{Z}_{100}$ on the range $0 \leq n \leq 25$. (with $n^2$ also representing values on the same negative range since they will become positive on squaring anyway)

Finally, observe this handy-dandy chart on the next page:

\begin{align*}
n^2 &\equiv m \pmod {100} \\
1^2 &\equiv 1 \pmod {100} \\
2^2 &\equiv 4 \pmod {100} \\
3^2 &\equiv 9 \pmod {100} \\
4^2 &\equiv 16 \pmod {100} \\
5^2 &\equiv 25 \pmod {100} \\
6^2 &\equiv 36 \pmod {100} \\
7^2 &\equiv 49 \pmod {100} \\
8^2 &\equiv 64 \pmod {100} \\
9^2 &\equiv 81 \pmod {100} \\
10^2 &\equiv 100 \pmod {100} \\
11^2 &\equiv 121 \pmod {100} \\
12^2 &\equiv 144 \pmod {100} \\
13^2 &\equiv 169 \pmod {100} \\
14^2 &\equiv 196 \pmod {100} \\
15^2 &\equiv 225 \pmod {100} \\
16^2 &\equiv 256 \pmod {100} \\
17^2 &\equiv 289 \pmod {100} \\
18^2 &\equiv 324 \pmod {100} \\
19^2 &\equiv 361 \pmod {100} \\
20^2 &\equiv 400 \pmod {100} \\
21^2 &\equiv 441 \pmod {100} \\
22^2 &\equiv 484 \pmod {100} \\
23^2 &\equiv 529 \pmod {100} \\
24^2 &\equiv 576 \pmod {100} \\
25^2 &\equiv 625 \pmod {100} \\
\end{align*}

\item Integrating this method with an implementation of Fermat's factorization method would speed it up by determining which combinations of patterns of squares would give \textbf{N} and then skipping the $a$ values which cannot possibly result in a correct answer. $b$ could also be checked similarly.

\end{enumerate} 

\end{homeworkProblem}

\end{document}
