
\documentclass{article}

\usepackage{fancyhdr}
\usepackage{extramarks}
\usepackage{amsmath}
\usepackage{amsthm}
\usepackage{amsfonts}
\usepackage{tikz}
\usepackage{amssymb}
\usepackage{stackrel}


\usetikzlibrary{automata,positioning}

%
% Basic Document Settings
%

\topmargin=-0.45in
\evensidemargin=0in
\oddsidemargin=0in
\textwidth=6.5in
\textheight=9.0in
\headsep=0.25in

\linespread{1.1}

\pagestyle{fancy}
\lhead{\hmwkAuthorName}
\chead{\hmwkClass\ (\hmwkClassInstructor\ \hmwkClassTime): \hmwkTitle}
\rhead{\firstxmark}
\lfoot{\lastxmark}
\cfoot{\thepage}

\renewcommand\headrulewidth{0.4pt}
\renewcommand\footrulewidth{0.4pt}

\setlength\parindent{0pt}

%
% Create Problem Sections
%

\newcommand{\enterProblemHeader}[1]{
    \nobreak\extramarks{}{Problem \arabic{#1} continued on next page\ldots}\nobreak{}
    \nobreak\extramarks{Problem \arabic{#1} (continued)}{Problem \arabic{#1} continued on next page\ldots}\nobreak{}
}

\newcommand{\exitProblemHeader}[1]{
    \nobreak\extramarks{Problem \arabic{#1} (continued)}{Problem \arabic{#1} continued on next page\ldots}\nobreak{}
    \stepcounter{#1}
    \nobreak\extramarks{Problem \arabic{#1}}{}\nobreak{}
}

\setcounter{secnumdepth}{0}
\newcounter{partCounter}
\newcounter{homeworkProblemCounter}
\setcounter{homeworkProblemCounter}{1}
\nobreak\extramarks{Problem \arabic{homeworkProblemCounter}}{}\nobreak{}

%
% Homework Problem Environment
%
% This environment takes an optional argument. When given, it will adjust the
% problem counter. This is useful for when the problems given for your
% assignment aren't sequential. See the last 3 problems of this template for an
% example.
%
\newenvironment{homeworkProblem}[1][-1]{
    \ifnum#1>0
        \setcounter{homeworkProblemCounter}{#1}
    \fi
    \section{Problem \arabic{homeworkProblemCounter}}
    \setcounter{partCounter}{1}
    \enterProblemHeader{homeworkProblemCounter}
}{
    \exitProblemHeader{homeworkProblemCounter}
}

%
% Homework Details
%   - Title
%   - Due date
%   - Class
%   - Section/Time
%   - Instructor
%   - Author
%

\newcommand{\hmwkTitle}{Exam\ \#2}
\newcommand{\hmwkDueDate}{December 4, 2017}
\newcommand{\hmwkClass}{Cryptography}
\newcommand{\hmwkClassTime}{Section 001}
\newcommand{\hmwkClassInstructor}{Karlsson, Christer H}
\newcommand{\hmwkAuthorName}{\textbf{Zachary Owen}}

%
% Title Page
%

\title{
    \vspace{2in}
    \textmd{\textbf{\hmwkClass:\ \hmwkTitle}}\\
    \normalsize\vspace{0.1in}\small{Due\ on\ \hmwkDueDate\ at 3:00pm}\\
    \vspace{0.1in}\large{\textit{\hmwkClassInstructor\ \hmwkClassTime}}
    \vspace{3in}
}

\author{\hmwkAuthorName}
\date{}

\renewcommand{\part}[1]{\textbf{\large Part \Alph{partCounter}}\stepcounter{partCounter}\\}

%
% Various Helper Commands
%


% Alias for the Solution section header
\newcommand{\solution}{\bigskip \textbf{\large Solution}}

\begin{document}

\maketitle

\pagebreak

\begin{homeworkProblem}
(15 pts) When we talk about attacks on RSA, we usually talk about factoring n into pq .  But no one has been able to prove that this is the only way to find $d$ and $e$ given $n$ .  For example we could try to find $\phi(n) = (p - 1)(q - 1)$ without first finding p and q .  Show that finding $\phi(n)$ is sufficient to factor $n$ and therefore enough to find $d$ .

\solution

\end{homeworkProblem}

\pagebreak

\begin{homeworkProblem}
(15 pts) Determine which of the following polynomials are irreducible over $\mathbb{Z}_2$ [ x ] 
\begin{itemize}
 \item $x^5 + x^4 + 1 $
 \item $x^5 + x^3 + 1 $
 \item $x^5 + x^4 + x^2 + 1 $
\end{itemize}
 
 The field $\mathbb{F}_{2^8}$ , can be constructed as $\cfrac{\mathbb{Z}_2[ x ]}{ x^8 + x^4 + x^3 + x + 1 }$
\begin{itemize}
\item  Compute $( x^5 + 1) - 1 $
 \item Compute $( x^4 + x^2) \times ( x^3 + x + 1)$
\end{itemize}

\solution

\end{homeworkProblem}

\pagebreak

\begin{homeworkProblem}
(15 pts)  My brother is color-blind, and we used to play snooker, if the balls had moved from their original positions he could not distinguished between a red and the green ball, as it is only the color that makes them non-identical.  He was often skeptical that I was actually potting the balls in the correct order.  I like to be able to prove to him that the two balls are in fact differently-colored.  At the same time , I do not want him to learn which is red and which is green.  Device a zero-knowledge protocol that allow me to prove that he really has two different colored balls in front of him.  He is allowed to hold, move and handle the balls, I am only allowed to look at them.

\solution

\end{homeworkProblem}

\pagebreak

\begin{homeworkProblem}
(15 pts)  Prove: An odd prime $p$ is expressible as a sum of two squares if and only if $p \equiv 1 \pmod 4 $

\solution

\end{homeworkProblem}

\pagebreak

\begin{homeworkProblem}
(15 pts) A common way of storing passwords on a computer is to use DES with a password as the key to encrypt a fixed plaintext (often just 000...0).  The ciphertext is then stored in a file.  When someone log in, the procedure is repeated and the ciphertexts are compared.  Why is this a better method than using the password as the plaintext and a fixed key?

\solution

\end{homeworkProblem}

\pagebreak

\begin{homeworkProblem}
(15 pts) You have received the following message: 
\begin{verbatim}
(949, 2750), (8513, 28089), (5513, 8421), 
(4769, 4261), (18352, 12856), (17914, 28599),
(25231, 9196), (3809, 5997), (1477, 19626),
(19108, 22326), (24966, 631), (3494, 5974),
(10256, 30308), (29093, 15082), (4223, 25106), 
(3595, 18546), (11325, 3588), (5632, 4912),
(18067, 13223), (21530, 3138), (30949, 16065),
(29784, 7987), (6385, 5955), (27338, 10405), 
(31715, 15969), (15815, 28055), (10462, 13371),
(4852, 28393), (1331, 30788), (18117, 28680),
(2472, 11786), (27548 , 22909), (21980, 28433),
(2154, 3440), (21504, 22036), (13651, 18061),
(10676, 26545), (30974, 23306), (14689, 8359) 
\end{verbatim}
It is an ElGamal ciphertext with the following parameters: 

\renewcommand{\labelenumi}{ }
\begin{enumerate}
\item $p = 31847 $
\item $\alpha = 5 $
\item $\beta = 18074 $
\end{enumerate}

and your private random integer was $a = 7899$

You also know that i n order to translate the plaintext back into ordinary English text, you need to know how alphabetic characters were “encoded” as elements in $\mathbb{Z}_n$.  Each element of $\mathbb{Z}_n$ represent three alphabetic characters as in the following example:
\begin{verbatim}
DOG → 3x26 2 + 14x26 +  6 =  2398 
CAT → 2x26 2 +  0x26 + 19 =  1371 
ZZZ → 25x26 2 + 25x26 + 25 = 17575 
\end{verbatim}
Decrypt the message (who sent it?) , also explain why it is good that the first element in each pair is not the same.

\solution

\end{homeworkProblem}

\pagebreak

\begin{homeworkProblem}
(15 pts)  Find all the prime numbers p such that 
\renewcommand{\labelenumi}{\alph{enumi})}
\begin{enumerate}
\item  $p| 2^p+1$
\item  $p| 2^p-1$
\end{enumerate}

\solution

\end{homeworkProblem}

\pagebreak

\begin{homeworkProblem}
(15 pts)  I have five nieces and nephews, and I want to share a secret $(M)$ with them, and when three of them are in agreement they should be able to ‘unlock’ it.  I pick a prime $(p)$ larger than number of nieces and nephews and the secret number, $p = 17$.

I calculate five pairs $( x_i ,y_i)$ where $y_i \equiv M + s_1 x_i + s_2 x_i^2 \pmod p$, and $s_1$ and $s_2$ are integers that only I know, and $x_1 , . . .  , x_5$ are distinct integers greater than 0.

Note that $f(0) \equiv M \pmod p$

I keep the polynomial secrete but I share $p$ and give each of them a $( x_i ,y_i)$ pair.  Three of them finally got together and agreed to try to solve my secret Lauren, Cohen and Kirian: $(1,8)$, $(3,10)$ and $(5,11)$ .  The trouble is they can’t agree on the math, so they ask you for help to solve this.  Calculate the \textit{Lagrange Interpolating Polynomial} and identify the secret $(M)$ .

\solution

\end{homeworkProblem}

\pagebreak

\end{document}
